\documentclass[
    letterpaper,paper=portrait,fleqn,
    DIV=16,fontsize=12pt,twoside=semi,
    parskip=full-,
    headings=standardclasses]
{scrartcl}

\usepackage{scrlayer-scrpage}
\clearpairofpagestyles
\ohead{\pagemark}

\usepackage{mathtools}
\usepackage[fullfamily,footnotefigures,swash,lf,mathtabular]{MinionPro}
\usepackage[eqno,enum,lineno]{tabfigures}

\usepackage{enumitem}
\usepackage[dvipsnames]{xcolor}
\usepackage{booktabs}
\usepackage{ragged2e}
\usepackage{dcolumn}
\usepackage{tikz,pgfplots,contour}
\usepackage{multicol}

\usetikzlibrary{patterns}
\usetikzlibrary{arrows,arrows.meta}

% Increase contour width
\contourlength{0.15em}
% Reduce vertical spacing around headings
\RedeclareSectionCommand[
  runin=false,
  beforeskip=0.5\baselineskip
]{section}
\RedeclareSectionCommand[
  beforeskip=0pt
]{paragraph}
% Allow display-mode math to break across pages
\allowdisplaybreaks[1]
% Set RaggedRight paragraph indent
\setlength{\RaggedRightParindent}{\parindent}
% Create additional column types
\newcolumntype{d}[1]{D{.}{.}{#1}}
\newcolumntype{R}{>{\raggedleft\arraybackslash}p{1in}}
% Bold enumerate numbering
\setlist[enumerate,1]{label=\bfseries\arabic*.}
\setlist[enumerate,2]{label=\bfseries(\alph*),nosep}
% Sections counted in alphabet and formatted as ``Parts''
\renewcommand*\thesection{\Alph{section}}
\renewcommand\sectionformat{Part~\thesection:\enskip}

\begin{document}
\RaggedRight
\thispagestyle{plain}

ECO 120-04 \\
Lucas Reddinger \\
Friday 18 November 2022 \hfill Your full name: \underline{\hspace{3.25in}}

\vspace{0.7\baselineskip}
\textbf{\LARGE Exam 2}
\vspace{0.3\baselineskip}

Please label proper units as appropriate.

Show your work to maximize partial credit.

\clearpage

\section{A two-good economy of kale and dental floss}

Consider an economy that produces two goods---kale and dental floss. The production possibility frontier of the economy is shown below.

\begin{tikzpicture}[x=4mm, y=4mm, >=latex]
\begin{axis}[
  width={0.6\linewidth},
  grid=both,
  grid style={line width=.1pt, draw=black!40!white},
  xmin=0,
  ymin=0,
  xmax=500,
  ymax=1200,
  xtick distance=100,
  ytick distance=100,
  xlabel={kale (tons)},
  ylabel={dental floss (km)},
  xticklabel style={yshift=-2pt},
  yticklabel style={xshift=-2pt},
  xlabel style={yshift=-4pt},
  ylabel style={yshift=6pt}
]

\addplot+ [
  mark=*, thick,
  visualization depends on=\thisrow{xshift} \as \xshift,
  point meta=explicit symbolic,
  nodes near coords,
  every node near coord/.style={above right, xshift=\xshift}
] table [meta=label,meta index=xshift] {
x y label xshift
    0   1000 \contour{white}{$A$} -5
    100  900 \contour{white}{$B$}  0
    200  700 \contour{white}{$C$}  0
    300  400 \contour{white}{$D$}  0
    400    0 \contour{white}{$E$}  0
};

\draw (axis cs:  350, 225) node[above right,blue]{\contour{white}{\textbf{\large PPF}}};
\node[label={225:{\contour{white}{$F$}}},circle,fill,inner sep=0pt,minimum size=2mm] at (axis cs:200,500) {};
\end{axis}
\end{tikzpicture}

\begin{enumerate}

\item What is the cost of producing $C$ instead of $A$?

\vfill

\item What is the benefit of producing $C$ instead of $A$?

\vfill

\item What is the cost of producing $D$ instead of $C$?

\vfill

\item What is the benefit of producing $D$ instead of $C$?

\vfill

\vspace{-2\baselineskip}

\clearpage

\item At $F$, what is the cost of producing an additional 200 km of dental floss?

\vfill

\item At $F$, what is the cost of producing an additional 400 km of dental floss?

\vfill

\item At $F$, what is the cost of producing an additional 100 tons of kale?

\vfill

\item At $F$, what is the cost of producing an additional 200 tons of kale?

\vfill

\item Is production efficient at $B$?

\vfill

\item Is production efficient at $F$?

\vfill

\item Is the cost of kale increasing, decreasing, or constant?

\vfill

\item Is the cost of dental floss increasing, decreasing, or constant?

\vfill

\vspace{-2\baselineskip}

\end{enumerate}

\clearpage

Suppose that in 2020, the economy produces $B$. In 2021, the economy produces $D$.

\begin{tabular}{ccc}
\toprule
Year & Price of kale (\$ per ton) & Price of dental floss (\$ per km) \\
\midrule
2020 & 2 &  8 \\
2021 & 3 & 10 \\
\bottomrule
\end{tabular}

Please use 2020 as the base year for any real GDP or CPI calculations.

\begin{enumerate}[resume]

\item What was the nominal GDP of the two-good economy in 2020?

\vfill

\item What was the nominal GDP of the two-good economy in 2021?

\vfill

\item What was the percent change in nominal GDP from 2020 to 2021?

\vfill

\item What was the real GDP of the two-good economy in 2020?

\vfill

\item What was the real GDP of the two-good economy in 2021?

\vfill

\item What was the percent change in real GDP from 2020 to 2021?

\vfill

\vspace{-2\baselineskip}

\clearpage

\item Based on your GDP calculations, what was the inflation rate from 2020 to 2021?

\vfill

\item What was the rate of inflation in the price of kale from 2020 to 2021?

\vfill

\item What was the rate of inflation in the price of dental floss from 2020 to 2021?

\vfill

\item Calculate a consumer price index for 2020.

\vfill

\item Calculate a consumer price index for 2021.

\vfill

\item Calculate the percentage change in this CPI from 2020 to 2021.

\vfill

\item You calculated economy-wide inflation using GDP and again using CPI. How much higher is inflation calculated from CPI than inflation calculated from GDP?

\vfill

\vspace{-2\baselineskip}

\clearpage

\end{enumerate}

\section{The La Crosse-Onalaska, WI-MN metropolitan statistical area}

Consider these actual labor force data from the Bureau of Labor Statistics:

\begin{tabular}{l*4{d{6.0}}}
\toprule
\multicolumn{5}{c}{Labor force data for the La Crosse-Onalaska, WI-MN MSA} \\
\midrule
& \multicolumn{4}{c}{Number of persons} \\
\cmidrule{2-5}
& \multicolumn{1}{c}{Aug.~2019} & \multicolumn{1}{c}{Aug.~2020} & \multicolumn{1}{c}{Aug.~2021} & \multicolumn{1}{c}{Aug.~2022} \\
\midrule
Employment & 74064 & 72257 & 74949 & 74048 \\
Unemployment & 2147 & 3779 & 2192 & 1904 \\
\bottomrule
\end{tabular}

\begin{enumerate}[resume]

\item What was the unemployment rate in Aug.~2019?

\vfill

\item What was the unemployment rate in Aug.~2020?

\vfill

\item What was the unemployment rate in Aug.~2021?

\vfill

\item By how many percentage points did the unemployment rate change from Aug.~2019 to Aug.~2020?

\vfill

\item By how many percentage points did the unemployment rate change from Aug.~2020 to Aug.~2021?

\vfill

\item Do you think more jobs opened in the year prior to Aug.~2020 or the year following Aug.~2020?

\vfill

\end{enumerate}

\vspace{-2\baselineskip}
\clearpage

\section{Multiple choice}

Please choose the best response.

\begin{enumerate}[resume]

\item In a two-good economy that produces $x$ and $y$, production is efficient if and only if some positive amount of good $x$ must be forgone to produce more good $y$.
\vspace{-8pt}
\begin{enumerate}
\item True
\item False
\end{enumerate}

\item If the price of an asset is expected to rise in the future:
\vspace{-8pt}
\begin{enumerate}
\item asset owners will be more willing to sell it now.
\item it will be more in demand today.
\item the price of the asset will fall today.
\item the market is irrational.
\end{enumerate}

\item A reason that does NOT explain why frictional unemployment exists is:
\vspace{-8pt}
\begin{enumerate}
\item that new jobs are continually being created.
\item that some old jobs are always being destroyed.
\item that new workers are always entering the labor market.
\item the minimum wage.
\end{enumerate}

\item When the unemployment rate is very low, most of it tends to be:
\vspace{-8pt}
\begin{enumerate}
\item cyclical.
\item frictional.
\item seasonal.
\item structural.
\end{enumerate}

\item When the demand for labor is falling and employers have committed to high wages, \rule{1cm}{0.15mm} unemployment will result.
\vspace{-8pt}
\begin{enumerate}
\item frictional
\item cyclical
\item permanent
\item structural
\end{enumerate}

\item Firms pay an efficiency wage because:
\vspace{-8pt}
\begin{enumerate}
\item it reduces the risk of losing the best workers.
\item it is required by law.
\item they don't have to offer health insurance if they pay efficiency wages.
\item it reduces the employee's income tax liability.
\end{enumerate}

\end{enumerate}

\begin{tikzpicture}[scale=0.5]
\draw[thick,<->] (0,10) node[below left]{wage $w$}--(0,0)--(10,0) node[below right]{$L$ (hours of labor)};
\draw[thick,red] (0,1)--(10,9) node[below]{$S$};
\draw[thick,blue] (0,9)--(9,0) node[above]{$D$};
\draw[dashed] (0,8) node[left]{$w_3$} --++ (10,0);
\draw[dashed] (0,4.5) node[left]{$w_2$} --++ (10,0);
\draw[dashed] (0,2) node[left]{$w_1$} --++ (10,0);
\draw[dashed] (1,0) node[below]{$L_1$} --++ (0,10);
\draw[dashed] (4.45,0) node[below]{$L_2$} --++ (0,10);
\draw[dashed] (7,0) node[below]{$L_3$} --++ (0,10);
\draw[dashed] (8.8,0) node[below]{$L_4$} --++ (0,10);
\end{tikzpicture}

\begin{enumerate}[resume]

\item What is the quantity of labor demanded at a binding minimum wage of $w_3$?
\vspace{-8pt}
\begin{multicols}{2}
\begin{enumerate}
\item $L_1$
\item $L_2$
\item $L_3$
\item $L_4$
\end{enumerate}
\end{multicols}

\item What is the quantity of labor supplied at a binding minimum wage of $w_3$?
\vspace{-8pt}
\begin{multicols}{2}
\begin{enumerate}
\item $L_1$
\item $L_2$
\item $L_3$
\item $L_4$
\end{enumerate}
\end{multicols}

\item The binding minimum wage of $w_3$ leads to surplus of labor of:
\vspace{-8pt}
\begin{multicols}{2}
\begin{enumerate}
\item $L_3 - L_1$.
\item $L_3 - L_2$.
\item $L_4 - L_1$.
\item $L_4 - L_2$.
\end{enumerate}
\end{multicols}

\end{enumerate}


\end{document}
