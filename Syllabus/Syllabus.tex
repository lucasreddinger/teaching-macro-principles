\documentclass[
    letterpaper,paper=portrait,fleqn,
    DIV=16,fontsize=11pt,twoside=semi,
    parskip=full-,
    headings=standardclasses]
{scrartcl}

\usepackage{scrlayer-scrpage}
\clearpairofpagestyles
\ohead{\pagemark}

\usepackage[fullfamily,swash,lf,mathtabular]{MinionPro}
\usepackage[eqno,enum,lineno]{tabfigures}

\usepackage[table,usenames,dvipsnames]{xcolor}
\usepackage{hyperref}
\usepackage{ragged2e,booktabs,enumitem}

\ihead[Last updated 14 December 2022.]{ECO 120-04, Fall 2022}

% Reduce vertical spacing around headings
\RedeclareSectionCommand[
  runin=false,afterindent=false,
  beforeskip=4pt,
  afterskip=0pt
]{section}
\RedeclareSectionCommand[
  runin=false,afterindent=false,
  beforeskip=0pt,
  afterskip=-3pt
]{subsection}
\RedeclareSectionCommand[
  runin=false,afterindent=false,
  beforeskip=-4pt,
  afterskip=-8pt
]{subsubsection}
\RedeclareSectionCommand[
  runin=true,afterindent=false,
  beforeskip=0pt,
  afterskip=12pt
]{paragraph}

\begin{document}
\RaggedRight
\thispagestyle{plain}

\phantom{.}

{\centering
\textbf{\huge Global Macroeconomics}\\
\vspace{\baselineskip}
{\Large ECO 120-04, Fall 2022}\\
\vspace{\baselineskip}
{\Large University of Wisconsin-La Crosse}\\
\vspace{2\baselineskip}
}

\section*{Instructor}

Name: Lucas Reddinger \\
Email: \href{mailto:lreddinger@uwlax.edu}{\nolinkurl{lreddinger@uwlax.edu}} \\
Office: 2120 Wittich Hall

Please email me for an appointment and feel free to drop-in anytime.

\section*{Course description}

Introduction to the functioning of the world economy. Applications of economic principles to domestic and international problems with an introduction to economic systems, economic thought, and economic history around the world. General topics include the economics of international exchange rates, global macroeconomics, international monetary systems, and economic development.

\section*{Learning outcomes}

\begin{enumerate}[nolistsep,noitemsep]

\item Foundational skills for macroeconomic analysis

\begin{enumerate}[nolistsep,noitemsep,label*=\arabic*.]
\item Apply the model of the production possibilities curve to illustrate the concepts of scarcity, choice, opportunity cost, and economic growth.
\item Use the supply and demand model to predict price and quantity outcomes for markets for products and services.
\item Use the supply and demand model for currencies to predict changes in exchange rates.
\item Define macroeconomic measures of production, prices, inflation, and employment.  Students will be able to explain how each is measured and evaluate usefulness and limitations for each measure.
\item Compare and explain international differences in macroeconomic outcomes of production, prices, inflation, and employment.
\end{enumerate}

\item Short-run fluctuations in the business cycle

\begin{enumerate}[nolistsep,noitemsep,label*=\arabic*.]
\item Apply the model of aggregate demand and aggregate supply to predict and demonstrate how changes in spending decisions and production costs affect real GDP and price level in the short run and long run.
\item Apply the model of aggregate demand and aggregate supply to predict and demonstrate how international influences affect real GDP and price level in the short run and long run.
\item Apply the model of aggregate demand and aggregate supply to current international economic and political issues.
\item Apply the model of aggregate demand and aggregate supply to evaluate the impact of fiscal and monetary policy on real GDP and price level in the short run and long run.
\end{enumerate}

\item Factors affecting long-run economic well being

\begin{enumerate}[nolistsep,noitemsep,label*=\arabic*.]
\item Predict how savings, investment decisions, and policies influence capital stock and long-run production possibilities.
\item Describe factors that may influence economic growth and use these to explain international difference in growth and development.
\end{enumerate}

\end{enumerate}

\section*{Textbook and materials}

\emph{Macroeconomics in Modules, 5th Edition}, by Paul Krugman and Robin Wells, 2022.

I will provide any other required materials.

\section*{Course outline}

This curriculum may change; I will give students timely notice in class and on Canvas.

\subsection*{Part A: Economic principles}

\subsubsection*{Unit 1: Principles}

Module 1: First Principles \\
Module 2: Models and the Production Possibility Frontier \\
% Module 3: Comparative Advantage and Trade \\
Module 4: Circular Flow \\
Module 25: The Financial System

\subsubsection*{Unit 2: Supply and Demand}

Module 5: Demand \\
Module 6: Supply and Equilibrium \\
Module 7: Changes in Equilibrium \\
Module 24: Market for Loanable Funds

\subsection*{Part B: Macroeconomic measurement}

\subsubsection*{Unit 3: Output}

Module 13: Introduction to Macroeconomics \\
Module 14: National Accounts and the Gross Domestic Product \\
Module 15: Interpreting Real Gross Domestic Product

\subsubsection*{Unit 4: Unemployment}

Module 16: Defining Unemployment \\
Module 17: Categories of Unemployment

\subsubsection*{Unit 5: Inflation}

Module 18: The Costs of Inflation \\
Module 19: Measuring Inflation

\subsection*{Part C: Macroeconomics}

\subsubsection*{Unit 6: Aggregate Demand and Aggregate Supply}

Module 30: Aggregate Demand \\
Module 31: Aggregate Supply \\
Module 32: The AD-AS Model

\subsubsection*{Unit 7: Fiscal policy}

Module 33: Fiscal Policy Basics \\
Module 34: Fiscal Policy and the Multiplier \\
Module 35: Budget Deficits and the Public Debt

\subsubsection*{Unit 8: Monetary policy}

Module 39: The Federal Reserve and Monetary Policy \\
Module 40: The Money Market \\
Module 41: Monetary Policy and the Interest Rate

\subsubsection*{Unit 9: International macroeconomics}

Module 46: Capital Flows and the Balance of Payments \\
Module 47: The Foreign Exchange Market

\section*{Grade composition and scale}

Your overall score will be the largest score calculated using either weighting scheme:

\vspace{0.2\baselineskip}
\begin{tabular}{llcc}
\toprule
Component               & Date               & \multicolumn{2}{c}{Weight} \\
\cmidrule{3-4}
 & & Scheme A & Scheme B \\
\midrule
Assignments and quizzes & \emph{ad hoc}                       & 25\% & 40\%  \\
Exam 1                  & October 14                          & 25\% & 20\%  \\
Exam 2                  & November 18                         & 25\% & 20\%  \\
Final Exam              & December 21, 7:45~a.m.~--~9:45~a.m. & 25\% & 20\%  \\
\bottomrule
\end{tabular}
\vspace{0.2\baselineskip}

Improved performance on the final exam may (at least partially) supplant prior exam scores.

When assigning final grades, I carefully review the final exams and may adjust (i.e., improve) the score-to-grade correspondence for the entire class so that students who have demonstrated competency earn at least a ``C.''

Then based on your maximally-weighted score (i.e., using either Scheme A or B that yields the best overall score), you will receive the following letter grade or better (I only adjust this scale to give better grades):

\vspace{0.2\baselineskip}
\begin{tabular}{lccccccc}
\toprule
Score & below 60\% & 60 -- 69\% & 70 -- 78\% & 79 -- 82\% & 83 -- 88\% & 89 -- 92\% & 93 -- 100\% \\
Grade & F & D & C & BC & B & AB & A \\
\bottomrule
\end{tabular}

\section*{Late assignments}

I penalize late work incrementally,

\begin{itemize}[nosep]
\item for being submitted after the deadline,
\item for the amount of time it is late, and
\item if it is submitted after solutions are posted.
\end{itemize}
Accordingly, \emph{to maximize credit earned, all late work should be submitted expediently.}

\paragraph{Example} Suppose an assignment is due on Monday, November 14. After grading and returning assignments, I post solutions on Friday, November 18. You submit your assignment on Monday, November 21. You may lose one point for the assignment being late, one-half points for each day it is late, and two points for being submitted after solutions have been posted.

Students with excuses approved by the Office of Student Life (e.g., COVID-19 quarantine) or the ACCESS Center (e.g., disability-related issues) will not be penalized.

You may submit two assignments one week late without penalty. Late penalties will be refunded at the end of the semester, once all assignment scores are settled. Assignments more than one-week late can receive a partial, one-week refund. Of all your late assignments, I will issue the two largest refunds.

\section*{Missed exams and quizzes}

I will generally offer an alternative time to write an exam or quiz to students with excuses approved by the Office of Student Life (e.g., COVID-19 quarantine) or ACCESS (e.g., disability-related issues). In rare circumstances, I may instead choose to proportionally scale up the weight of the other scores of the same type.

\section*{Format}

This is a face-to-face course. I may ask you to participate through Canvas, the online learning-management system. I may also provide materials through Canvas.

\section*{Attendance and participation}

Although your attendance does not directly affect your grade, I will survey attendance on occasion.

\section*{Expectations for graded work}

I will provide feedback or scores on graded work before another assignment of a similar format is due. I will generally return work that requires individual feedback or scores within 14 days of its due date. I will notify you if I am unable to return your work within this time frame and will identify a revised return date. If you submit work after the due date, I may not return it within this time frame.

I will return your graded work in compliance with FERPA regulations.

\clearpage

\section*{COVID-19 health statement}

All UWL students are encouraged to be vaccinated against COVID-19. All students are required to be masked in classrooms and other indoor campus communal spaces. Campus-wide mask guidance may change during the semester. Students with COVID-19 symptoms or reason to believe they were exposed to SARS-CoV-2 should consult with a health professional, such as the UWL Student Health Center (608-785-8558), regardless of their vaccination status. Students who are ill or engaging in self-quarantine at the direction of a health professional should not attend class. Students in this situation will not be required to provide formal documentation and will not be penalized for absences. However, students should:

\begin{itemize}[nosep]
\item notify instructors in advance of the absence and provide the instructor with an idea of how long the absence may last, if possible. 
\item keep up with classwork if able. 
\item submit assignments electronically. 
\item work with instructors to either reschedule or remotely complete exams, labs, and other academic activities.
\item consistently communicate their status to the instructor during the absence.
\end{itemize}

Instructors have an obligation to provide reasonable accommodation for completing course requirements to students adversely affected by COVID-19. This policy relies on honor, honesty, and mutual respect between instructors and students. Students are expected to report the reason for absence truthfully and instructors are expected to trust the word of their students. UWL codes of conduct and rules for academic integrity apply to COVID-19 situations. Students may be advised by their instructor or academic advisor to consider a medical withdrawal depending on the course as well as timing and severity of illness, and students should work with the Office of Student Life if pursuing a medical withdrawal.

\section*{Academic success and overall health}

At UWL, we support your academic success and overall health. We know that students often experience a range of stressors that can impact learning and well-being. If you or someone you know is experiencing mental health concerns, or could benefit from effective academic strategies, there are free and confidential resources available to enrolled students through the Counseling \& Testing Center (CTC). To learn more, visit  \href{https://www.uwlax.edu/counseling-testing/}{CTC's website} or call 608-785-8073.

\section*{UWL Syllabus Policy information and statements}

UWL encourages students to know the campus' important policies related to COVID-19 health statement, academic integrity \& misconduct, religious accommodations, sexual misconduct, student concern procedures, students with disabilities, and veterans \& active military personnel. These policies and statements can be found on  \href{https://www.uwlax.edu/info/syllabus/}{the Syllabus Information website}.

Individual instructors will articulate course requirements and any additional policies in the course syllabus and/or on a Canvas site associated with the course. UWL also encourages students to take advantage of the campus's many and varied student success resources; a listing is found on \href{https://www.uwlax.edu/info/student-success}{the UWL Student Success website}.

\section*{UWL policies \& supports}

\subsection*{Course access}

Access to course materials in Canvas may cease after the term ends. If you wish to archive materials for your personal records or portfolio you should do so as you progress through the course. As a general rule, you should always save local copies of course-related work. To avoid disasters, you should also save important files to external media or cloud storage.

\subsection*{Inclusive excellence}

\href{https://www.uwlax.edu/chancellor/mission/}{UWL's core values} include ``Diversity, equity, and the inclusion and engagement of all people in a safe campus climate that embraces and respects the innumerable different perspectives found within an increasingly integrated and culturally diverse global community.'' If you are not experiencing my class in this manner, please come talk to me about your experiences so I can try to adjust the course if possible.

\subsection*{Name and pronouns}

I will do my best to address you by a preferred name or gender pronoun that you have identified. Please advise me of this preference early in the semester so that I may make appropriate changes to my records. UWL has a  \href{https://www.uwlax.edu/records/preferred-name/}{preferred name policy} and \href{https://www.uwlax.edu/pride-center/}{UWL's Pride Center} is available for additional assistance.

\subsection*{PRO@UWL (Progress Report Online via Navigate) and Student Success Policy}

If I notice that you are experiencing difficulties early in the semester (e.g., low assignment scores or limited participation), I may provide you feedback through Navigate, UWL's success system, and you will receive notification indicating that I have entered feedback. I encourage you to meet with me and utilize helpful campus resources listed on \href{https://www.uwlax.edu/info/student-success}{the UWL Student Success website}.

\subsection*{Student Evaluation of Instruction (SEI)}

UWL conducts student evaluations electronically. Approximately 2 weeks prior to the conclusion of a course, you will receive an email at your UWL email address directing you to complete an evaluation for each of your courses. In-class time will be provided for students to complete the evaluation in class. Electronic reminders will be sent if you do not complete the evaluation. The evaluation will include numerical ratings and, depending on the department, may provide options for comments. The university takes student feedback very seriously and the information gathered from student evaluations is more valuable when a larger percentage of students complete the evaluation. Please be especially mindful to complete the surveys.

\subsection*{Academic services and resources at UWL}

UWL has an array of supports for your success, and the website below makes sure you know about all of them. Please take the time to investigate the resources available via \href{https://www.uwlax.edu/info/student-success}{the Student Success website}. If you would like help finding what you need, please do talk with me.

\subsection*{Technical Support}

For tips and information about Canvas visit \href{https://www.uwlax.edu/info/canvas/students/}{the UWL Canvas Guide for students}; this site also links to the 24/7 Canvas support. If you have Canvas login issues or need general computer help, contact \href{https://www.uwlax.edu/its/client-services-and-support/eagle-help-desk/}{the Eagle Help Desk}.

\end{document}
