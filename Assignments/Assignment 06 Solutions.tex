\documentclass[
    letterpaper,paper=portrait,fleqn,
    DIV=16,fontsize=12pt,twoside=semi,
    parskip=full-,
    headings=standardclasses]
{scrartcl}

\usepackage{scrlayer-scrpage}
\clearpairofpagestyles
\ohead[Last updated 16 November 2022.]{\pagemark}

\usepackage{mathtools}
\usepackage[fullfamily,footnotefigures,swash,lf,mathtabular]{MinionPro}
\usepackage[eqno,enum,lineno]{tabfigures}

\usepackage{enumitem}
\usepackage[dvipsnames]{xcolor}
\usepackage{booktabs}
\usepackage{ragged2e}
\usepackage{dcolumn}
\usepackage{tikz}
\usepackage{pgfplots}
\usepackage[outline]{contour}
\usepackage{mdframed}

\usetikzlibrary{patterns}
\usetikzlibrary{arrows,arrows.meta}
\contourlength{0.15em}

% Reduce vertical spacing around headings
\RedeclareSectionCommand[
  runin=false,afterindent=false,
  beforeskip=0.5\baselineskip,
  afterskip=0.5\baselineskip
]{section}
\RedeclareSectionCommand[
  runin=false,afterindent=false,
  beforeskip=0pt
]{paragraph}
% Allow display-mode math to break across pages
\allowdisplaybreaks[1]
% Set RaggedRight paragraph indent
\setlength{\RaggedRightParindent}{\parindent}
% Create additional column types
\newcolumntype{d}[1]{D{.}{.}{#1}}
\newcolumntype{R}{>{\raggedleft\arraybackslash}p{1in}}
% Bold enumerate numbering
\setlist[enumerate,1]{label=\bfseries\arabic*.}
%\setlist{itemsep=\baselineskip,parsep=0\baselineskip}
% Sections counted in alphabet and formatted as ``Parts''
\renewcommand*\thesection{\Alph{section}}
\renewcommand\sectionformat{Part~\thesection:\enskip}

\newmdenv[leftline=true,rightline=false,topline=false,bottomline=false,linewidth=2pt]{solution}

\begin{document}
\RaggedRight
\thispagestyle{plain}

ECO 120-04 \\
Lucas Reddinger \\
Wednesday 2 November 2022

\vspace{0.7\baselineskip}
\textbf{\LARGE Assignment 6 Solutions}

\section{BLS definitions\label{sec:bls-definitions}}

The Bureau of Labor Statistics (BLS) surveys people aged 16 or older and applies these definitions:
\begin{itemize}[nosep]
\item \emph{Employed}: Those who performed any work (including self-employment or on a family farm).
\item \emph{Unemployed}: Those who had no employment, were available for employment, and looked for a job in the preceding four weeks.
\item \emph{Labor force}: The sum of employed and unemployed people.
\item \emph{Unemployment rate}: The number unemployed as a percent of the labor force.
\item \emph{Labor force participation rate}: The labor force as a percent of the population.
\end{itemize}

\section{Entering and exiting the labor force}

For this section, please
\begin{itemize}[nosep]
\item use four digits after the decimal in your answers,
\item specify units, such as percent (\%) or percentage points (p.p.) as applicable, and
\item use the data in the following table and the BLS definitions above.
\end{itemize}

\begin{tabular}{ld{3.0}}
\toprule
\multicolumn{2}{c}{BLS data for the U.S.~in February 2020} \\
\midrule
Classification & \multicolumn{1}{l}{People, millions} \\
\midrule
Employed & 158.8 \\ 
Unemployed & 5.8 \\
Working-age population & 259.6 \\
\bottomrule
\end{tabular}

\clearpage

\begin{enumerate}
\item Write the formula for \emph{the labor force participation rate} as a function of \emph{employment}, \emph{unemployment}, and \emph{the working-age population}.

\begin{solution}
\vspace{-1.0\baselineskip}
\begin{align*}
\text{labor force participation rate} = \frac{ \text{employment} + \text{unemployment} }{ \text{the working-age population} }.
\end{align*}
\end{solution}

\item What was the labor force participation rate in February 2020?

\begin{solution}
\vspace{-1.0\baselineskip}
\begin{align*}
\text{labor force participation rate} = \frac{ 158.8 \text{ M people} + 5.8 \text{ M people} }{ 259.6 \text{ M people} } = \frac{164.6}{259.6} = 0.6341 = 63.41\%.
\end{align*}
\end{solution}

\item What was the unemployment rate in February 2020?

\begin{solution}
\vspace{-1.0\baselineskip}
\begin{align*}
\text{unemployment rate} &= \frac{ \text{unemployment} }{ \text{the labor force} } = \frac{ \text{unemployment} }{ \text{employment} + \text{unemployment} }.\\
\text{unemployment rate} &= \frac{ 5.8 \text{ M people} }{ 158.8 \text{ M people} + 5.8 \text{ M people} } = \frac{5.8}{164.6} = 0.0352 = 3.52\%.
\end{align*}
\end{solution}

\item Suppose that in March 2020, 1 million unemployed people quit looking for a job, and the number of employed people stayed the same as in February 2020.

\begin{enumerate}
\item What would be the unemployment rate in March 2020?

\begin{solution}
\vspace{-1.0\baselineskip}
\begin{align*}
\text{unemployment rate} &= \frac{ 4.8 \text{ M people} }{ 158.8 \text{ M people} + 4.8 \text{ M people} } = \frac{4.8}{163.6} = 0.0293 = 2.93\%.
\end{align*}
\end{solution}

\item What was the change in the unemployment rate from February 2020 to March 2020?

\begin{solution}
\vspace{-1.0\baselineskip}
\begin{align*}
\text{change} &= x_\text{final} - x_\text{initial} \\
\text{change in unemp rate} &= \text{unemp rate}_\text{2020-Mar} - \text{unemp rate}_\text{2020-Feb} \\
&= 2.93\% - 3.52\% = -0.59 \text{ p.p.}
\end{align*}

Suppose we mistakenly wrote that the unemployment rate changed by $-0.59\%$; we wrote units of percent instead of percentage points (p.p.). Then $\text{unemp rate}_\text{2020-Mar} = (100\% - 0.59\%) \times \text{unemp rate}_\text{2020-Feb} = (1 - 0.0059) \times 3.52\% = 3.49\%$. This isn't the true value.

Here, a decrease in the unemployment rate of $0.59$ percentage points instead conveys this: the prior rate, $3.52\%$ declined by $0.59$ percentage \emph{points}. Doing this simpler arithmetic, we find that $\text{unemp rate}_\text{2020-Mar} = 3.52\% - 0.59 \text{ p.p.} = 2.93 \%$, which is the true value.
\end{solution}

\item Why did the unemployment rate change without any change in employment?

\begin{solution}
People who switch between employment and unemployment obviously affect the rate.

In this case, however, unemployed people decided to leave the labor force altogether, which affected the unemployment rate.
\end{solution}

\end{enumerate}

\item \emph{Now instead suppose} that in March 2020, the number of unemployed people remained the same as in February 2020, while 5 million people quit their jobs to be stay-at-home parents.

\begin{enumerate}

\item In this case, what would be the unemployment rate in March 2020? Please specify units.

\begin{solution}
\vspace{-1.0\baselineskip}
\begin{align*}
\text{unemp rate}_\text{2020-Mar} &= \frac{ 5.8 \text{ M people} }{ 153.8 \text{ M people} + 5.8 \text{ M people} } = 4.10\%
\end{align*}
\end{solution}

\item What was the change in the unemployment rate from February 2020 to March 2020? Please specify units.

\begin{solution}
\vspace{-1.0\baselineskip}
\begin{align*}
\text{change in unemp rate} = 4.10\% - 3.52\% = 0.57 \text{ p.p.}
\end{align*}
\end{solution}

\item Why did the unemployment rate change without any change in unemployment?

\begin{solution}
Employed people left the labor force. As a result, unemployed people constitute a greater share of the labor force.
\end{solution}

\end{enumerate}

\end{enumerate}

\section{Production and output}

For this section, please use the fictional data below. \emph{Please specify units for each answer.}

\begin{tabular}{l*5{d{3.0}}}
\toprule
Year & \multicolumn{2}{c}{Carrots} & \multicolumn{2}{c}{Onions}\\
\cmidrule(r){2-3} \cmidrule(l){4-5}
& \multicolumn{1}{c}{Price (\$/ton)} & \multicolumn{1}{c}{Output (tons)} & \multicolumn{1}{c}{Price (\$/ton)} & \multicolumn{1}{c}{Output (tons)} \\
\midrule
2019 & 2 & 10 & 3 & 15 \\
2020 & 3 &  8 & 4 & 11 \\
\bottomrule
\end{tabular}

\begin{enumerate}

\item What was nominal GDP in 2019?

\begin{solution}
\vspace{-1.0\baselineskip}
\begin{align*}
\text{nom. gdp}_\text{2019} = \frac{\$2}{\text{ton}} \times 10 \text{ tons} + \frac{\$3}{\text{ton}} \times 15 \text{ tons} = \$65
\end{align*}
\end{solution}

\item What was nominal GDP in 2020?

\begin{solution}
\vspace{-1.0\baselineskip}
\begin{align*}
\text{nom. gdp}_\text{2020} = \frac{\$3}{\text{ton}} \times 8 \text{ tons} + \frac{\$4}{\text{ton}} \times 11 \text{ tons} = \$68
\end{align*}
\end{solution}

\item What was the percent change in nominal GDP from 2019 to 2020?

\begin{solution}
\vspace{-1.0\baselineskip}
\begin{align*}
\text{percent change} &= \frac{ x_\text{final} - x_\text{initial} }{ x_\text{initial} } \\
\text{\% change in nom. gdp} &= \frac{\$68 - \$65}{\$65} = 4.62\%
\end{align*}
\end{solution}

\item For this question, please use 2019 as the base year.

\begin{enumerate}

\vspace{-6pt}
\item What was real GDP in 2019?

\begin{solution}
$\$65$
\end{solution}

\vspace{-6pt}
\item What was real GDP in 2020?

\begin{solution}
$\$49$
\end{solution}

\vspace{-6pt}
\item What was the percent change in real GDP from 2019 to 2020?

\begin{solution}
$-24.61\%$
\end{solution}

\end{enumerate}

\vspace{-6pt}
\item For this question, please use 2020 as the base year.

\begin{enumerate}

\vspace{-6pt}
\item What was real GDP in 2019?

\begin{solution}
$\$90$
\end{solution}

\vspace{-6pt}
\item What was real GDP in 2020?

\begin{solution}
$\$68$
\end{solution}

\vspace{-6pt}
\item What was the percent change in real GDP from 2019 to 2020?

\begin{solution}
$-24.44\%$
\end{solution}

\end{enumerate}

\vspace{-6pt}
\item Why is the percent change in nominal GDP so different than the percent change in real GDP?

\begin{solution}
The price of carrots increased by $50\%$. The price of onions increased by $33\%$. This inflation made the change in output larger in nominal terms than it was in real terms.
\end{solution}

\end{enumerate}

\end{document}
