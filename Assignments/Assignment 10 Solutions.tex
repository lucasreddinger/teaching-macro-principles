\documentclass[
    letterpaper,paper=portrait,fleqn,
    DIV=16,fontsize=12pt,twoside=semi,
    parskip=full-,
    headings=standardclasses]
{scrartcl}

\usepackage{scrlayer-scrpage}
\clearpairofpagestyles
\ohead{\pagemark}

\usepackage{mathtools}
\usepackage[fullfamily,footnotefigures,swash,lf,mathtabular]{MinionPro}
\usepackage[eqno,enum,lineno]{tabfigures}

\usepackage{enumitem}
\usepackage[dvipsnames]{xcolor}
\usepackage[capitalise,nameinlink]{cleveref}
\usepackage{booktabs}
\usepackage{ragged2e}
\usepackage{dcolumn}
\usepackage{tikz}
\usepackage{pgfplots}
\usepackage[outline]{contour}
\usepackage{mdframed}

\usetikzlibrary{patterns}
\usetikzlibrary{arrows,arrows.meta}
\contourlength{0.15em}

% Reduce vertical spacing around headings
\RedeclareSectionCommand[
  runin=false,afterindent=false,
  beforeskip=0.5\baselineskip,
  afterskip=0.5\baselineskip
]{section}
\RedeclareSectionCommand[
  runin=false,afterindent=false,
  beforeskip=0pt
]{paragraph}
% Allow display-mode math to break across pages
\allowdisplaybreaks[1]
% Set RaggedRight paragraph indent
\setlength{\RaggedRightParindent}{\parindent}
% Create additional column types
\newcolumntype{d}[1]{D{.}{.}{#1}}
\newcolumntype{R}{>{\raggedleft\arraybackslash}p{1in}}
% Bold enumerate numbering
\setlist[enumerate,1]{label=\bfseries\arabic*.,ref={\arabic*}}
%\setlist{itemsep=\baselineskip,parsep=0\baselineskip}
% Sections counted in alphabet and formatted as ``Parts''
\renewcommand*\thesection{\Alph{section}}
\renewcommand\sectionformat{Part~\thesection:\enskip}
\crefname{section}{part}{parts}
\crefname{enumi}{question}{questions}

\newmdenv[leftline=true,rightline=false,topline=false,bottomline=false,linewidth=2pt]{solution}

\begin{document}
\RaggedRight
\thispagestyle{plain}

ECO 120-04 \\
Lucas Reddinger \\
Friday 9 December 2022

\vspace{0.7\baselineskip}
\textbf{\LARGE Assignment 10 Solutions}

Suppose that the U.S.~economy is initially in long-run equilibrium, producing its potential output $Y_P$, as depicted directly below.

\begin{tikzpicture}[scale=0.6]
\draw[thick,<->] (0,10) node[below left,label={[align=right]left:Aggregate\\price level\\ }] {$P$} --(0,0)--(18,0) node[below left,label=right:Real GDP]{$Y$};
\draw[very thick,blue] (1,6.5) --(12,1) node[right]{$\text{AD}_1$};
\draw[very thick,orange] (9,0) node[below]{$Y_P$}--(9,8) node[above]{LRAS};
\draw[very thick,red] (7,1)--(15,7) node[right]{$\text{SRAS}_1$};
\draw[dashed] (0,2.5) node[left]{$P^*_1$} --(9,2.5) ;
\end{tikzpicture}

A multinational effort to sanction Russian oil and natural gas results in a negative shock to short-run aggregate supply from $\text{SRAS}_1$ to $\text{SRAS}_2$ as depicted below. U.S.~economic output falls from $Y_P$ to $Y^*_2$, and the aggregate price level rises from $P^*_1$ to $P^*_2$. This results in a period of \emph{stagflation}---lower economic output coupled with inflation.

\begin{tikzpicture}[scale=0.6]
\draw[thick,<->] (0,10) node[below left,label={[align=right]left:Aggregate\\price level\\ }] {$P$} --(0,0)--(18,0) node[below left,label=right:Real GDP]{$Y$};
\draw[very thick,blue] (1,6.5) --(12,1) node[right]{$\text{AD}_1$};
\draw[very thick,orange] (9,0) node[below]{$Y_P$}--(9,8) node[above]{LRAS};
\draw[very thick,red] (7,1)--(15,7) node[right]{$\text{SRAS}_1$};
\draw[dashed] (0,2.5) node[left]{$P^*_1$} --(9,2.5) node[right,xshift=6pt]{$E_1$};
\draw[very thick,red] (2,1) --(11,7.75) node[right]{$\text{SRAS}_2$};
\draw[dashed] (0,4) node[left]{$P^*_2$} --(6,4) node[right,xshift=6pt]{$E_2$} --(6,0)node[below]{$Y^*_2$};
\draw[line width=2pt,->,red] (14,6.75)--(10.5,6.75);
\end{tikzpicture}


\begin{enumerate}

\item \label{maintain-output} Please illustrate how fiscal policy could be used to maintain economic growth. \\ {\footnotesize Hint: Reproduce the second graph here. Illustrate a change in fiscal policy that restores output to $Y_P$.}

\begin{solution}
To maintain economic growth, we would want to achieve actual output equal to potential output. Thus we would want to shift the AD curve until $Y_P$ is achieved with $\text{SRAS}_2$.

\begin{tikzpicture}[scale=0.6]
\draw[thick,<->] (0,10) node[below left,label={[align=right]left:Aggregate\\price level\\ }] {$P$} --(0,0)--(18,0) node[below left,label=below:Real GDP]{$Y$};
\draw[very thick,blue] (5,4.5)--(12,1) node[right]{$\text{AD}_1$};
\draw[very thick,orange] (9,0) node[below]{$Y_P$}--(9,8) node[above]{LRAS};
\draw[very thick,red] (7,1)--(15,7) node[right]{$\text{SRAS}_1$};
\draw[dashed] (0,2.5) node[left]{$P^*_1$}--(9,2.5) node[right,xshift=6pt]{};
\draw[very thick,red] (5,3.2)--(11,7.75) node[right]{$\text{SRAS}_2$};
\draw[dashed] (0,4) node[left]{$P^*_2$}--(6,4) node[right,xshift=6pt]{} --(6,0)node[below]{$Y^*_2$};

\draw[very thick,blue] (5,8.25)--(17,2.5) node[right]{$\text{AD}_3$};
\draw[dashed] (0,6.25) node[left]{$P^*_3$}--(9,6.25) node[right,xshift=6pt]{};
\node[below,yshift=-14pt] at (9,0) {$Y^*_3$};

\draw[line width=2pt,->,blue] (12,2.5)--(15,2.5);
\end{tikzpicture}
\end{solution}

\item \label{maintain-prices} Please illustrate how fiscal policy could be used to stabilize prices. \\ {\footnotesize Hint: Reproduce the second graph here. Illustrate a change in fiscal policy that restores the price level to $P^*_1$.}

\begin{solution}
To stabilize the price level, we would want to achieve P equal to $P^*_1$. Thus we would want to shift the AD curve until $P^*_1$ is achieved with $\text{SRAS}_2$.

\begin{tikzpicture}[scale=0.6]
\draw[thick,<->] (0,10) node[below left,label={[align=right]left:Aggregate\\price level\\ }] {$P$} --(0,0)--(18,0) node[below left,label=below:Real GDP]{$Y$};
\draw[very thick,blue] (1,6.5) --(12,1) node[right]{$\text{AD}_1$};
\draw[very thick,orange] (9,0) node[below]{$Y_P$}--(9,8) node[above]{LRAS};
\draw[very thick,red] (7,1)--(15,7) node[right]{$\text{SRAS}_1$};
\draw[dashed] (4,2.5) --(9,2.5) node[right,xshift=6pt]{};
\draw[very thick,red] (2,1) --(11,7.75) node[right]{$\text{SRAS}_2$};
\draw[dashed] (0,4) node[left]{$P^*_2$} --(6,4) node[right,xshift=6pt]{} --(6,0)node[below]{$Y^*_2$};

\draw[very thick,blue] (1,4) node[above]{$\text{AD}_3$} --(6.6,1.2);
\draw[dashed] (0,2.5) node[left,xshift=-16pt]{$P^*_3=$} --(4,2.5) node[above,xshift=-1pt]{} --(4,0) node[below]{$Y^*_3$};
\node[left] at (0,2.5) {$P^*_1$};

\draw[line width=2pt,->,blue] (4,4.5)--(2,4.5);
\end{tikzpicture}
\end{solution}

\item Can fiscal policy remedy stagflation? Please explain with a complete sentence.  \\ {\footnotesize Hint: Can a change in fiscal policy restore \emph{both} output to $Y_P$ \emph{and} the price level to $P^*_1$?}

\begin{solution}
Stagflation refers to output stagnation coupled with inflation. We see stagflation at $E_2$ relative to $E_1$ because $Y^*_2<Y_P$ (output is decreased) and $P^*_2>P^*_1$ (the price level is increased).

In \cref{maintain-output}, the proposed policy restores economic output ($Y^*_3=Y_P$), but it worsens inflation ($P^*_3 \gg P^*_1$). In \cref{maintain-prices}, the proposed policy restores the price level ($P^*_3=P^*_1$), but it further depresses output ($Y^*_3 \ll Y_P$).

We conclude that by shifting the aggregate demand curve, higher output can be achieved (but with higher prices) or lower prices can be achieved (but with lower inflation).

A shift of the aggregate demand curve \emph{cannot both} lower the price level and \emph{also} raise output.
\end{solution}

\end{enumerate}

\end{document}