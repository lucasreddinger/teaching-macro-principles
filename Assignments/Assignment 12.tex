\documentclass[
    letterpaper,paper=portrait,fleqn,
    DIV=16,fontsize=12pt,twoside=semi,
    parskip=full-,
    headings=standardclasses]
{scrartcl}

\usepackage{scrlayer-scrpage}
\clearpairofpagestyles
\ohead{\pagemark}

\usepackage{mathtools}
\usepackage[fullfamily,footnotefigures,swash,lf,mathtabular]{MinionPro}
\usepackage[eqno,enum,lineno]{tabfigures}
\usepackage{pifont}

\usepackage{enumitem}
\usepackage[dvipsnames]{xcolor}
\usepackage[capitalise,nameinlink]{cleveref}
\usepackage{booktabs}
\usepackage{ragged2e}
\usepackage{dcolumn}
\usepackage{tikz}

% Reduce vertical spacing around headings
\RedeclareSectionCommand[
  runin=false,afterindent=false,
  beforeskip=0.5\baselineskip,
  afterskip=0.5\baselineskip
]{section}
\RedeclareSectionCommand[
  runin=false,afterindent=false,
  beforeskip=0pt
]{paragraph}
% Allow display-mode math to break across pages
\allowdisplaybreaks[1]
% Set RaggedRight paragraph indent
\setlength{\RaggedRightParindent}{\parindent}
% Create additional column types
\newcolumntype{d}[1]{D{.}{.}{#1}}
\newcolumntype{R}{>{\raggedleft\arraybackslash}p{1in}}
% Bold enumerate numbering
\setlist[enumerate,1]{label=\bfseries\arabic*.,ref={\arabic*}}
%\setlist{itemsep=\baselineskip,parsep=0\baselineskip}
% Sections counted in alphabet and formatted as ``Parts''
\renewcommand*\thesection{\Alph{section}}
\renewcommand\sectionformat{Part~\thesection:\enskip}
\crefname{section}{part}{parts}
\crefname{enumi}{question}{questions}

\begin{document}
\RaggedRight
\thispagestyle{plain}

ECO 120-04 \\
Lucas Reddinger \\
Friday 2 December 2022 \hfill Your full name: \underline{\hspace{3.25in}}

\vspace{0.7\baselineskip}
\textbf{\LARGE Assignment 12: Aggregate supply and demand}
\vspace{0.3\baselineskip}

\emph{Due Wednesday 7 December.} Please submit hardcopy at the beginning of class (11:00 a.m.), or if you prefer, under the door of Wimberly Hall 339C by 10:50 a.m.

{\centering

\vspace{-0.5\baselineskip}
\Pisymbol{MinionPro-Extra}{121}
\vspace{-0.5\baselineskip}

}

Please present a model of the Russian economy that relates output to an aggregate price level.

\begin{enumerate}

\item Please graph the Russian economy in long-run equilibrium. Mark all curves with subscript ``0'' as well as output ($Y_0$) and price level ($P_0$). Be sure to also show potential output on your graph.

\vfill

\item On Tuesday the U.S.~Treasury Secretary Janet Yellen reported that the Russian economy will experience inflation of about 20\% and a 10 to 15\% decline in output.

Suppose that these changes are the result of a short-run shift of \emph{one} curve in your model above. In the space directly below, please write which curve shifts and in which direction.

\vspace{3.0\baselineskip}

\clearpage

\item On your graph above, please draw this shift, labeling new curves with subscript ``1,'' the new output level as $Y_1$, and the new price level as $P_1$.

\item Is $Y_1$ above, below, or equal to potential output?

\vspace{3.0\baselineskip}

\item Please describe the output gap at $Y_1$ using a mathematical statement. (For example, ``the output gap $<100\%$'' or ``the output gap $<0$.'')

\vspace{4.0\baselineskip}

\item Consider possible explanations of why that particular curve with subscript ``1'' might shift in that direction. Give one explanation consistent with current global economic events.

\emph{For example}, if this problem were about the U.S. economy during the pandemic, you might say that aggregate demand increased. A possible explanation of why aggregate demand might increase during the pandemic is that households were given monetary transfers---stimulus payments---which increased aggregate demand.

\vfill

\item Please explain how this economy will return to long-run equilibrium. What specifically changes? What curve shifts as a result?

\vspace{4.0\baselineskip}

\item On your graph above, please draw this shift, labeling new curves with subscript ``2,'' the new output level as $Y_2$, and the new price level as $P_2$.

\item Is $Y_2$ above, below, or equal to potential output?

\vspace{3.0\baselineskip}

\item Please describe the output gap at $Y_2$ using a mathematical statement.

\vspace{2.0\baselineskip}

\end{enumerate}

\end{document}
