\documentclass{assignment}

\course{ECO 120-04}
\name{Lucas Reddinger}
\date{Friday 2 December 2022}
\doctitle{Assignment 12: Aggregate supply and demand}

\begin{document}
\RaggedRight

\beginassignment{}

\emph{Due Wednesday 7 December.} Please submit hardcopy at the beginning of class (11:00 a.m.), or if you prefer, under the door of Wimberly Hall 339C by 10:50 a.m.

\ornamentalrule

Please present a model of the Russian economy that relates output to an aggregate price level.

\begin{enumerate}

\item Please graph the Russian economy in long-run equilibrium. Mark all curves with subscript ``0'' as well as output ($Y^*_0$) and price level ($P^*_0$). Be sure to also show potential output on your graph.

\vfill

\item On Tuesday the U.S.~Treasury Secretary Janet Yellen reported that the Russian economy will experience inflation of about 20\% and a 10 to 15\% decline in output.

Suppose that these changes are the result of a short-run shift of \emph{one} curve in your model above. In the space directly below, please write which curve shifts and in which direction.

\vspace{3.0\baselineskip}

\clearpage

\item On your graph above, please draw this shift, labeling new curves with subscript ``1,'' the new output level as $Y^*_1$, and the new price level as $P^*_1$.

\item Is $Y^*_1$ above, below, or equal to potential output?

\vspace{3.0\baselineskip}

\item Please describe the output gap at $Y^*_1$ using a mathematical statement. (For example, ``the output gap $<100\%$'' or ``the output gap $<0$.'')

\vspace{4.0\baselineskip}

\item Consider possible explanations of why that particular curve with subscript ``1'' might shift in that direction. Give one explanation consistent with current global economic events.

\emph{For example}, if this problem were about the U.S. economy during the pandemic, you might say that aggregate demand increased. A possible explanation of why aggregate demand might increase during the pandemic is that households were given monetary transfers---stimulus payments---which increased aggregate demand.

\vfill

\item Please explain how this economy will return to long-run equilibrium. What specifically changes? What curve shifts as a result?

\vspace{4.0\baselineskip}

\item On your graph above, please draw this shift, labeling new curves with subscript ``2,'' the new output level as $Y^*_2$, and the new price level as $P^*_2$.

\item Is $Y^*_2$ above, below, or equal to potential output?

\vspace{3.0\baselineskip}

\item Please describe the output gap at $Y^*_2$ using a mathematical statement.

\vspace{2.0\baselineskip}

\end{enumerate}

\end{document}
