\documentclass[
    letterpaper,paper=portrait,fleqn,
    DIV=16,fontsize=12pt,twoside=semi,
    parskip=full-,
    headings=standardclasses]
{scrartcl}

\usepackage{scrlayer-scrpage}
\clearpairofpagestyles
\ohead{\pagemark}

\usepackage{mathtools}
\usepackage[fullfamily,footnotefigures,swash,lf,mathtabular]{MinionPro}
\usepackage[eqno,enum,lineno]{tabfigures}

\usepackage{enumitem}
\usepackage[dvipsnames]{xcolor}
\usepackage{booktabs}
\usepackage{ragged2e}
\usepackage{dcolumn}
\usepackage{tikz}

% Reduce vertical spacing around headings
\RedeclareSectionCommand[
  runin=false,
  beforeskip=0.5\baselineskip
]{section}
\RedeclareSectionCommand[
  beforeskip=0pt
]{paragraph}
% Allow display-mode math to break across pages
\allowdisplaybreaks[1]
% Set RaggedRight paragraph indent
\setlength{\RaggedRightParindent}{\parindent}
% Create additional column types
\newcolumntype{d}[1]{D{.}{.}{#1}}
\newcolumntype{R}{>{\raggedleft\arraybackslash}p{1in}}
% Bold enumerate numbering
\setlist[enumerate,1]{label=\bfseries\arabic*.}
%\setlist{itemsep=\baselineskip,parsep=0\baselineskip}
% Sections counted in alphabet and formatted as ``Parts''
\renewcommand*\thesection{\Alph{section}}
\renewcommand\sectionformat{Part~\thesection:\enskip}

\begin{document}
\RaggedRight
\thispagestyle{plain}

ECO 120-04 \\
Lucas Reddinger \\
Wednesday 14 September 2022 \hfill Your full name: \underline{\hspace{3.25in}}

\vspace{0.7\baselineskip}
\textbf{\LARGE Assignment 1: Production possibility frontier}
\vspace{0.3\baselineskip}

Please work on this with your assigned group today during class. If you don't finish during class, please complete the assignment outside of class with other students or individually.

\emph{Due Friday 16 September.} Please submit hardcopy at the beginning of class (11:00 a.m.), or if you prefer, under the door of Wimberly Hall 339C by 10:50 a.m.

\section*{Iowa agriculture}

Suppose that Iowa produces only two goods, corn and soybeans. When production is efficient, Iowa can produce these combinations of corn and soybeans:

{\footnotesize\begin{tabular}{RR}
\toprule
\multicolumn{2}{c}{Production of bushels, millions} \\
\midrule
\multicolumn{1}{c}{Corn} & \multicolumn{1}{c}{Soybeans} \\
\midrule
    0 & 2,000 \\
1,500 & 1,750 \\
3,000 & 1,500 \\
4,500 & 1,000 \\
6,000 & 0 \\
\bottomrule
\end{tabular}}

\begin{enumerate}

\item Draw the production possibility frontier for Iowa's two-good economy, using linear segments (straight lines) to connect the points in the table. Label the feasible production possibilities and the efficient production possibilities. Also label the axes and the production possibility frontier.

\begin{center}
\begin{tikzpicture}[x=4mm, y=4mm]
\clip(3,3) rectangle (32,17);
\draw[step=4mm, very thin, black!20!white] (0,0) grid (60,120);
\draw[step=20mm, thick, black!40!white] (0,0) grid (60,120);
\end{tikzpicture}
\end{center}

\item Suppose Iowa is producing 1,000 million bushels of corn and 1,750 million bushels of soybeans. Mark this point labeled $B$ on your graph above. Is $B$ efficient? Please explain with a sentence.


\vfill

\item Now instead suppose that Iowa is producing 1,500 million bushels of corn and 1,750 million bushels of soybeans---label this $C$. Is $C$ efficient? At this particular combination of production, what is the cost of the 1,500 million bushels of corn being produced?

\vfill

\item Now suppose that Iowa is producing 3,750 million bushels of corn and 1,250 million bushels of soybeans---label this $D.$ Is $D$ efficient? At this particular combination of production, what is the cost of producing the last million bushels of corn?

\vfill

\item Let's again consider the point $B$ from above. At those particular levels of production, what is the cost of producing an additional 500 million bushels of corn?

\vfill

\item Use your preceding answer to explain what constitutes efficient production in a sentence or two.

\vfill

\vspace{-4\baselineskip}

\end{enumerate}
\end{document}
