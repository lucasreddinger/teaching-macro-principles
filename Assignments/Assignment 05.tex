\documentclass[
    letterpaper,paper=portrait,fleqn,
    DIV=16,fontsize=12pt,twoside=semi,
    parskip=full-,
    headings=standardclasses]
{scrartcl}

\usepackage{scrlayer-scrpage}
\clearpairofpagestyles
\ohead{\pagemark}

\usepackage{mathtools}
\usepackage[fullfamily,footnotefigures,swash,lf,mathtabular]{MinionPro}
\usepackage[eqno,enum,lineno]{tabfigures}

\usepackage{enumitem}
\usepackage[dvipsnames]{xcolor}
\usepackage{booktabs}
\usepackage{ragged2e}
\usepackage{dcolumn}
\usepackage{tikz}

% Reduce vertical spacing around headings
\RedeclareSectionCommand[
  runin=false,
  beforeskip=0.5\baselineskip
]{section}
\RedeclareSectionCommand[
  beforeskip=0pt
]{paragraph}
% Allow display-mode math to break across pages
\allowdisplaybreaks[1]
% Set RaggedRight paragraph indent
\setlength{\RaggedRightParindent}{\parindent}
% Create additional column types
\newcolumntype{d}[1]{D{.}{.}{#1}}
\newcolumntype{R}{>{\raggedleft\arraybackslash}p{1in}}
% Bold enumerate numbering
\setlist[enumerate,1]{label=\bfseries\arabic*.}
%\setlist{itemsep=\baselineskip,parsep=0\baselineskip}
% Sections counted in alphabet and formatted as ``Parts''
\renewcommand*\thesection{\Alph{section}}
\renewcommand\sectionformat{Part~\thesection:\enskip}

\begin{document}
\RaggedRight
\thispagestyle{plain}

ECO 120-04 \\
Lucas Reddinger \\
Friday 30 September 2022 \hfill Your full name: \underline{\hspace{3.25in}}

\vspace{0.7\baselineskip}
\textbf{\LARGE Assignment 5: Supply and demand}
\vspace{0.3\baselineskip}

\emph{Due Wednesday 5 October.} Please submit hardcopy at the beginning of class (11:00 a.m.), or if you prefer, under the door of Wimberly Hall 339C by 10:50 a.m.

\section{The U.S.~market for used cars}

\begin{center}
\begin{tikzpicture}[x=4mm, y=4mm]
\clip(3,3) rectangle (32,22);
\draw[step=4mm, very thin, black!20!white] (0,0) grid (60,120);
\draw[step=20mm, thick, black!40!white] (0,0) grid (60,120);
\end{tikzpicture}
\vspace{18pt}
\end{center}

\begin{enumerate}

\item Depict the U.S.~market for used cars in equilibrium with a supply and demand model. Label the curves as $S$ and $D$, equilibrium price as $p^*_1$, equilibrium quantity as $q^*_1$, and axes as appropriate.

\item Suppose supply chain disruptions significantly reduce the availability of cars. Appropriately depict the new market-clearing equilibrium (e.g., prices, quantities, curves) using subscripts of 2.

\item What type of market is this---a factor market or a product market?

\vfill

\item What entities does the demand curve represent?

\vfill

\vspace{-1\baselineskip}
\clearpage

\item What entities does the supply curve represent?

\vfill

\item Do you know whether $q^*_2$ is higher than, lower than, or the same as $q^*_1$?

\vfill

\item Do you know whether $p^*_2$ is higher than, lower than, or the same as $p^*_1$?

\vfill

\end{enumerate}

\section{The U.S.~labor market}

\begin{center}
\begin{tikzpicture}[x=4mm, y=4mm]
\clip(3,3) rectangle (32,22);
\draw[step=4mm, very thin, black!20!white] (0,0) grid (60,120);
\draw[step=20mm, thick, black!40!white] (0,0) grid (60,120);
\end{tikzpicture}
\vspace{18pt}
\end{center}

\begin{enumerate}

\item Depict the U.S.~labor market in equilibrium with a supply and demand model. Label the curves as $S$ and $D$, equilibrium price as $w^*_1$, equilibrium quantity as $L^*_1$, and axes as appropriate.

\item Suppose that due to COVID-19 cases declining, more people want to work. Appropriately shift any curves, labeled with subscripts of 2.

\item Because the demand for consumer goods is strong, firms want to hire more workers. Appropriately shift any curves, labeled with subscripts of 3.

\item Given this tandem of changes to supply and demand, please label the new equilibrium wage as $w^*_3$ and equilibrium quantity of labor traded as $L^*_3$.

\clearpage

\item What type of market is this---a factor market or a product market?

\vfill

\item What entities does the demand curve represent?

\vfill

\item What entities does the supply curve represent?

\vfill

\item Considering the two changes in tandem, has labor traded increased, decreased, or stayed the same? Or do you not have enough information to determine this overall effect? That is, do you know whether $L^*_3$ is higher than, lower than, or the same as $L^*_1$?

\vfill

\item Considering the two changes in tandem, has the wage increased, decreased, or stayed the same? Or do you not have enough information to determine this overall effect? That is, do you know whether $w^*_3$ is higher than, lower than, or the same as $w^*_1$?

\vfill

\end{enumerate}

\vspace{-1\baselineskip}
\clearpage

\section{The U.S.~loanable funds market}

\emph{Please complete this part independently to obtain valuable feedback on your personal progress.}

\begin{center}
\begin{tikzpicture}[x=4mm, y=4mm]
\clip(3,3) rectangle (32,22);
\draw[step=4mm, very thin, black!20!white] (0,0) grid (60,120);
\draw[step=20mm, thick, black!40!white] (0,0) grid (60,120);
\end{tikzpicture}
\vspace{18pt}
\end{center}

\begin{enumerate}

\item Depict the loanable funds market in equilibrium with a supply and demand model. Label the curves as $S$ and $D$, equilibrium rate as $r^*_1$, equilibrium quantity as $Q^*_1$, and axes as appropriate.

\item Suppose that many investors divest from Russia, moving funds to the U.S. Appropriately depict the new market-clearing equilibrium (e.g., interest rates, quantities, curves) using subscripts of 2.

\item Has the market-clearing interest rate increased, decreased, or stayed the same?

\vfill

\item What type of market is this---a factor market or a product market?

\vfill

\item What entities does the demand curve represent?

\vfill

\item What entities does the supply curve represent?

\vfill

\vspace{-1\baselineskip}

\end{enumerate}

\end{document}
