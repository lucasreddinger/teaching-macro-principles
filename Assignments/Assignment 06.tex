\documentclass[
    letterpaper,paper=portrait,fleqn,
    DIV=16,fontsize=12pt,twoside=semi,
    parskip=full-,
    headings=standardclasses]
{scrartcl}

\usepackage{scrlayer-scrpage}
\clearpairofpagestyles
\ohead{\pagemark}

\usepackage{mathtools}
\usepackage[fullfamily,footnotefigures,swash,lf,mathtabular]{MinionPro}
\usepackage[eqno,enum,lineno]{tabfigures}

\usepackage{enumitem}
\usepackage[dvipsnames]{xcolor}
\usepackage{booktabs}
\usepackage{ragged2e}
\usepackage{dcolumn}
\usepackage{tikz}

% Reduce vertical spacing around headings
\RedeclareSectionCommand[
  runin=false,
  beforeskip=0.5\baselineskip
]{section}
\RedeclareSectionCommand[
  beforeskip=0pt
]{paragraph}
% Allow display-mode math to break across pages
\allowdisplaybreaks[1]
% Set RaggedRight paragraph indent
\setlength{\RaggedRightParindent}{\parindent}
% Create additional column types
\newcolumntype{d}[1]{D{.}{.}{#1}}
\newcolumntype{R}{>{\raggedleft\arraybackslash}p{1in}}
% Bold enumerate numbering
\setlist[enumerate,1]{label=\bfseries\arabic*.}
%\setlist{itemsep=\baselineskip,parsep=0\baselineskip}
% Sections counted in alphabet and formatted as ``Parts''
\renewcommand*\thesection{\Alph{section}}
\renewcommand\sectionformat{Part~\thesection:\enskip}

\begin{document}
\RaggedRight
\thispagestyle{plain}

ECO 120-04 \\
Lucas Reddinger \\
Monday 24 October 2022 \hfill Your full name: \underline{\hspace{3.25in}}

\vspace{0.7\baselineskip}
\textbf{\LARGE Assignment 6: Employment and output}
\vspace{0.3\baselineskip}

\emph{Due Friday 28 October.} Please submit hardcopy at the beginning of class (11:00 a.m.), or if you prefer, under the door of Wimberly Hall 339C by 10:50 a.m.

\section{BLS definitions\label{sec:bls-definitions}}

The Bureau of Labor Statistics (BLS) surveys people aged 16 or older and applies these definitions:
\begin{itemize}
\item \emph{Employed}: Those who performed any work (including self-employment or on a family farm).
\item \emph{Unemployed}: Those who had no employment, were available for employment, and looked for a job in the preceding four weeks.
\item \emph{Labor force}: The sum of employed and unemployed people.
\item \emph{Unemployment rate}: The number unemployed as a percent of the labor force.
\item \emph{Labor force participation rate}: The labor force as a percent of the population.
\end{itemize}

\section{Entering and exiting the labor force}

For this section, please
\begin{itemize}[nosep]
\item use four digits after the decimal in your answers,
\item specify units, such as percent (\%) or percentage points (p.p.) as applicable, and
\item use the data in the following table and the BLS definitions above.
\end{itemize}

\begin{tabular}{ld{3.0}}
\toprule
Classification & \multicolumn{1}{l}{People (millions)} \\
\midrule
Employed & 158.8 \\ 
Unemployed & 5.8 \\
Working-age population & 259.6 \\
\bottomrule
\end{tabular}

\clearpage

\begin{enumerate}
\item Write the formula for \emph{the labor force participation rate} as a function of \emph{employment}, \emph{unemployment}, and \emph{the working-age population}.

\vfill

\item What was the labor force participation rate in February 2020?
\vfill
\item What was the unemployment rate in February 2020?
\vfill
\item Suppose that in March 2020, 1 million unemployed people quit looking for a job, and the number of employed people stayed the same as in February 2020.

\begin{enumerate}
\item What would be the unemployment rate in March 2020?
\vfill
\item What was the change in the unemployment rate from February 2020 to March 2020?
\vfill
\item Why did the unemployment rate change without any change in employment?
\vfill
\end{enumerate}

\clearpage

\item \emph{Now instead suppose} that in March 2020, the number of unemployed people remained the same as in February 2020, while 5 million people quit their jobs to be stay-at-home parents.

\begin{enumerate}
\item In this case, what would be the unemployment rate in March 2020? Please specify units.
\vfill
\item What was the change in the unemployment rate from February 2020 to March 2020? Please specify units.
\vfill
\item Why did the unemployment rate change without any change in unemployment?
\vfill
\vfill
\end{enumerate}
\end{enumerate}

\section{Production and output}

For this section, please use the fictional data below. \emph{Please specify units for each answer.}

\begin{tabular}{l*5{d{3.0}}}
\toprule
Year & \multicolumn{2}{c}{Carrots} & \multicolumn{2}{c}{Onions}\\
\cmidrule(r){2-3} \cmidrule(l){4-5}
& \multicolumn{1}{c}{Price (\$/ton)} & \multicolumn{1}{c}{Output (tons)} & \multicolumn{1}{c}{Price (\$/ton)} & \multicolumn{1}{c}{Output (tons)} \\
\midrule
2019 & 2 & 10 & 3 & 15 \\
2020 & 3 &  8 & 4 & 11 \\
\bottomrule
\end{tabular}

\begin{enumerate}
\item What was nominal GDP in 2019?
\vfill
\item What was nominal GDP in 2020?
\vfill
\item What was the percent change in nominal GDP from 2019 to 2020?
\vfill

\clearpage

\item For this question, please use 2019 as the base year.
\begin{enumerate}
\item What was real GDP in 2019?
\vfill
\item What was real GDP in 2020?
\vfill
\item What was the percent change in real GDP from 2019 to 2020?
\vfill
\end{enumerate}
\item For this question, please use 2020 as the base year.
\begin{enumerate}
\item What was real GDP in 2019?
\vfill
\item What was real GDP in 2020?
\vfill
\item What was the percent change in real GDP from 2019 to 2020? 
\vfill
\end{enumerate}
\item Why is the percent change in nominal GDP so different than the percent change in real GDP?
\vfill
\end{enumerate}

\end{document}
