\documentclass[
    letterpaper,paper=portrait,fleqn,
    DIV=16,fontsize=12pt,twoside=semi,
    parskip=full-,
    headings=standardclasses]
{scrartcl}

\usepackage{scrlayer-scrpage}
\clearpairofpagestyles
\ohead{\pagemark}

\usepackage{mathtools}
\usepackage[fullfamily,footnotefigures,swash,lf,mathtabular]{MinionPro}
\usepackage[eqno,enum,lineno]{tabfigures}

\usepackage{enumitem}
\usepackage[dvipsnames]{xcolor}
\usepackage[capitalise,nameinlink]{cleveref}
\usepackage{booktabs}
\usepackage{ragged2e}
\usepackage{dcolumn}
\usepackage{tikz}
\usepackage{pgfplots}
\usepackage[outline]{contour}
\usepackage{mdframed}

\usetikzlibrary{patterns}
\usetikzlibrary{arrows,arrows.meta}
\contourlength{0.15em}

% Reduce vertical spacing around headings
\RedeclareSectionCommand[
  runin=false,afterindent=false,
  beforeskip=0.5\baselineskip,
  afterskip=0.5\baselineskip
]{section}
\RedeclareSectionCommand[
  runin=false,afterindent=false,
  beforeskip=0pt
]{paragraph}
% Allow display-mode math to break across pages
\allowdisplaybreaks[1]
% Set RaggedRight paragraph indent
\setlength{\RaggedRightParindent}{\parindent}
% Create additional column types
\newcolumntype{d}[1]{D{.}{.}{#1}}
\newcolumntype{R}{>{\raggedleft\arraybackslash}p{1in}}
% Bold enumerate numbering
\setlist[enumerate,1]{label=\bfseries\arabic*.,ref={\arabic*}}
%\setlist{itemsep=\baselineskip,parsep=0\baselineskip}
% Sections counted in alphabet and formatted as ``Parts''
\renewcommand*\thesection{\Alph{section}}
\renewcommand\sectionformat{Part~\thesection:\enskip}
\crefname{section}{part}{parts}
\crefname{enumi}{question}{questions}

\newmdenv[leftline=true,rightline=false,topline=false,bottomline=false,linewidth=2pt]{solution}

\begin{document}
\RaggedRight
\thispagestyle{plain}

ECO 120-04 \\
Lucas Reddinger \\
Wednesday 21 September 2022

\vspace{0.7\baselineskip}
\textbf{\LARGE Assignment 1 Solutions}

\section*{Iowa agriculture}

Suppose that Iowa produces only two goods, corn and soybeans. When production is efficient, Iowa can produce these combinations of corn and soybeans:

{\footnotesize\begin{tabular}{lrrrrr}
\toprule
Good & \multicolumn{5}{c}{Production of bushels, millions} \\
\midrule
Corn     & 0 & 1,500 & 3,000 & 4,500 & 6,000 \\
Soybeans & 2,000 & 1,750 & 1,500 & 1,000 & 0 \\
\bottomrule
\end{tabular}}

\begin{enumerate}

\item Draw the production possibility frontier for Iowa's two-good economy, using linear segments (straight lines) to connect the points in the table. Label the feasible production possibilities and the efficient production possibilities. Also label the axes and the production possibility frontier.

\begin{solution}
\begin{center}
\begin{tikzpicture}[x=4mm, y=4mm, >=latex, scale=0.8]
\begin{axis}[width={0.8\linewidth},grid=both,
  grid style={line width=.1pt, draw=black!40!white},
  xmin=0,
  ymin=0,
  xtick=data,
  ytick distance=250,
  xlabel={corn (million bushels)},
  ylabel={soybeans (million bushels)},
  xticklabel style={yshift=-2pt},
  yticklabel style={xshift=-2pt},
  xlabel style={yshift=-4pt},
  ylabel style={yshift=6pt}
]
  
\begin{scope}
\clip (0,0) -- (axis cs:   0, 2000) -- (axis cs:1500, 1750) -- (axis cs:3000, 1500) -- (axis cs:4500, 1000) -- (axis cs:6000,    0) -- cycle;
\fill[pattern color=green,pattern=north east lines] (0,0) rectangle (axis cs: 6000, 2000);
\node[label={[font=\large,text=green]0:{\contour{white}{\textbf{feasible set}}}}] at (axis cs:1750,750) {};
\end{scope}

\addplot+ [mark=*,line width=1mm,mark size=1mm] table {
x y
   0 2000
1500 1750
3000 1500
4500 1000
6000    0
};

\node[label={45:{\contour{white}{\textbf{\large $A$}}}},circle,fill,inner sep=1pt] at (axis cs:0,2000) {};
\node[label={210:{\contour{white}{\textbf{\large $B$}}}},circle,fill,inner sep=0pt,minimum size=3mm] at (axis cs:1000,1750) {};
\node[label={45:{\contour{white}{\textbf{\large $C$}}}},circle,fill,inner sep=1pt] at (axis cs:1500,1750) {};
\node[label={45:{\contour{white}{\textbf{\large $D$}}}},circle,fill,inner sep=1pt,minimum size=3mm] at (axis cs:3750,1250) {};
\draw[blue,line width=1mm,<-](axis cs:  4900, 800)--(axis cs:  5500, 1150) node[above,blue]{\contour{white}{\textbf{\large PPF (efficient set)}}};

\end{axis}
\end{tikzpicture}
\end{center}
\vspace{-0.7\baselineskip}
\end{solution}

\item Suppose Iowa is producing 1,000 million bushels of corn and 1,750 million bushels of soybeans. Mark this point labeled $B$ on your graph above. Is $B$ efficient? Please explain with a sentence.


\begin{solution}
$B$ is not efficient. For example, an additional 500 M.~bu.~of corn could be produced. Instead of increasing corn production by that amount, the Iowa economy could instead produce some additional corn and some additional soybeans.
\end{solution}

\item Now instead suppose that Iowa is producing 1,500 million bushels of corn and 1,750 million bushels of soybeans---label this $C$. Is $C$ efficient? At this particular combination of production, what is the cost of the 1,500 million bushels of corn being produced?

\begin{solution}
\emph{Complete answer:} $C$ is efficient. The cost of the 1,500 M.~bu.~of corn being produced at $C$ is 250 M.~bu.~of soybeans.

$C$ is efficient. It is one of the points that constitutes the PPF, as provided in the table.

If the economy did not produce those 1,500 M.~bu.~of corn, it would be producing no corn. This would allow the economy to instead produce 2,000 M.~bu.~of soybeans. (This new point is marked $A$ on the graph above.) At $C,$ the economy had already been producing 1,750 M.~bu.~of soybeans. Thus, by foregoing the 1,500 M.~bu.~of corn, the economy produces an additional 250 M.~bu.~of soybeans. Therefore the cost of the 1,500 M.~bu.~of corn being produced at point $C$ is 250 M.~bu.~of soybeans.
\end{solution}
\item Now suppose that Iowa is producing 3,750 million bushels of corn and 1,250 million bushels of soybeans---label this $D.$ Is $D$ efficient? At this particular combination of production, what is the cost of producing the last million bushels of corn?

\begin{solution}
\emph{Complete answer:} $D$ is efficient. Each million bushels of corn costs $1/3$ million bushels of soybeans.

$D$ is efficient; it is on the PPF. The cost of the last million bushels of corn can be found using the slope of the line segment on which $D$ lies. This slope is $$\frac{1000-1500\text{ M.~bu.~of soybeans}}{4500-3000\text{ M.~bu.~of corn}}=-1/3\text{ M.~bu.~of soybeans per 1 M.~bu.~of corn.}$$
Thus each million bushels of corn costs $1/3$ million bushels of soybeans on the PPF line segment in the neighborhood of point $D.$

Note that the cost is different on the other line segments, as the other line segments have different slopes.
\end{solution}

\clearpage

\item Let's again consider the point $B$ from above. At those particular levels of production, what is the cost of producing an additional 500 million bushels of corn?

\begin{solution}
\emph{Complete answer:} It costs nothing.

Producing an additional 500 million bushels of corn at $B$ costs nothing, as the economy will become exactly efficient. This change would move the economy from point $B$ to point $C,$ which is on the PPF.
\end{solution}

\item Use your preceding answer to explain what constitutes efficient production in a sentence or two.

\begin{solution}
When production is efficient, to gain additional production of one good, something must be forgone.
\end{solution}

\end{enumerate}

\end{document}
